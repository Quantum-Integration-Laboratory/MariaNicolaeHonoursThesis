\chapter{\label{ch:conclusion}Conclusion}

Hybrid microwave-optical quantum systems have a key role to play in efforts to produce large-scale quantum technology systems, as well as interoperating different quantum technologies. In particular, hybrid microwave-optical transducers and entangled photon pair generators are important tools for producing entanglement across lengths and inside volumes too large to cool to cryogenic temperatures. Atomic systems, in particular using rare-earth atoms in crystals, are an appealing platform with which to build such technologies. However, our understanding of these systems, and therefore our ability to optimise them, is incomplete. I have produced, and presented in this thesis, numerical models of such systems in realistic parameter spaces. These can be used as an aid to improve our understanding of these systems.

I produced a model that computes the output power of atomic ensemble based transduction, that accounts for four atomic energy levels, rather than just three as in prior modelling work. This allows the model to capture the effect of interference between the outputs of two atomic transitions, which can have a large effect on the efficiency of such a transducer. Even `three-level' transduction schemes often have a fourth level nearby, making interference effects broadly relevant. The formalism can readily be extended to systems with more than four levels and more than two interfering transitions.

This four-level transduction model shows qualitative agreement with experimental data, reproducing the essential features measured in a frequency sweep. There is still some quantitative discrepancy between model results and experimental data, which reflects uncertainty in the experimental parameters. Future work could involve better evaluating the model by either further refining the parameters for a better match with the experimental data used here, or identifying or producing another experimental dataset whose parameters relevant to the model are known with more certainty.

Separately, I produced a model for the photon pair generation rate of an atomic ensemble based biphoton generation process using a three-level system. The dynamical behaviours produced by the model are qualitatively explainable by and consistent with both theory and analogy to experimental results in other light-matter interaction systems. The results also demonstrated that the ODE problem in the model is stiff, which highlights that the results could be improved with a more suitable numerical approach than the RK4 method used. I theorised an efficient implementation of such a method.

Future work on this biphoton generation model could include implementing better numerical methods, as well as performing biphoton generation experiments to produce a dataset with which to validate the model. Separately, future work could be to build from this model to produce a model that predicts not only generation rate, but also the degree of entanglement generated within photon pairs. This is important because the degree of entanglement, not only the rate of photon pair generation, affects the rate at which quantum information can be transmitted. This could be done by replacing the mean-field approximation with a `Gaussian-field' approximation that incorporates the (co)variances of electromagnetic field operators in addition to expectation values.

In conclusion, the numerical models developed in this project show potential to be useful for understanding microwave-optical transducers and pair generators, thereby aiding the development of such technologies. However, they lack conclusive benchmarks against experimental results, such benchmarking being a prime avenue of future work.
